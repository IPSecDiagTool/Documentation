\chapter{Jenkins Buildserver}
Um repetitive Aufgaben möglichst zu vermindern haben wir Jenkins als \acs{CI} Server eingesetzt. Wir haben Jenkins so konfiguriert dass für jeden Commit, sowohl beim Programm als auch bei der LaTeX-Dokumentation, ein neuer Build-Job gestartet wird.
In diesem Kapitel werden die konkreten Konfigurationen dokumentiert.

\section{Systemvoraussetzungen}
Wir haben Jenkins auf dem HSR VPS mit einem Ubuntu 14.04 x64 betrieben. Dabei wurden folgende Abhängigkeiten vorinstalliert:

\begin{itemize}
  \item Jenkins 1.610
  \item libpcap0.8-dev
  \item go 1.4.2
  \item git 2.3.5
  \item texlive-full 2015
  \item golang lstlang0.sty
\end{itemize}

TODO: https://bitbucket.org/korfuri/golang-latex-listings/src/6334c5cbf471/lstlang0.sty
korrekt attributieren.

\section{IPSecDiagTool - Go Build Job}

\subsection{Job Einstellungen}
Die folgenden Einstellungen wurden beim Go Build Job verwendet:

\begin{table}[h]
\begin{tabular}{|l|l|}
\hline
\rowcolor[HTML]{C0C0C0} 
{\color[HTML]{000000} \textbf{Option}} & {\color[HTML]{000000} \textbf{Einstellung}}              \\ \hline
Discard Old Builds                          & Yes                                                      \\ \hline
Log Rotation                                & Max \# of builds to keep 5                               \\ \hline
SCM                                         & None                                                     \\ \hline
Build Trigger                               & None                                                     \\ \hline
Archive the artifcats                       & src/github.com/ipsecdiagtool/ipsecdiagtool/ipsecdiagtool \\ \hline
\end{tabular}
\end{table}

\subsection{Bash Script}
\begin{lstlisting}[language=bash]
#!/bin/sh
echo "Cleaning workspace"
if [ -d src ]; then
	rm -r *
fi

export GOROOT=/usr/local/go
export GOPATH="$WORKSPACE"
export PATH="$PATH:$GOROOT/bin:$GOPATH/bin"

echo "Downloading dependencies"
#sudo apt-get install libpcap-0.8-dev
go get code.google.com/p/gopacket
go get code.google.com/p/gopacket/pcap
go get golang.org/x/net/ipv4

echo "Moving to program directory"
cd src
mkdir -p github.com/ipsecdiagtool/
cd github.com/ipsecdiagtool
git clone https://github.com/IPSecDiagTool/IPSecDiagTool.git
mv IPSecDiagTool ipsecdiagtool
cd ipsecdiagtool

echo "Building program"
go version
go build
\end{lstlisting}



\section{Dokumentation - LaTex Build Job}

\subsection{Job Einstellungen}
Die folgenden Einstellungen wurden beim LaTeX Build Job verwendet:

\begin{table}[h]
\begin{tabular}{|l|l|}
\hline
\rowcolor[HTML]{C0C0C0} 
{\color[HTML]{000000} \textbf{Option}} & {\color[HTML]{000000} \textbf{Einstellung}}                                                                                           \\ \hline
Discard Old Builds                          & Yes                                                                                                                                   \\ \hline
Log Rotation                                & Max \# of builds to keep 5                                                                                                            \\ \hline
SCM                                         & \begin{tabular}[c]{@{}l@{}}Git\\     URL:      https://github.com/IPSecDiagTool/Documentation.git\\     Branch: */master\end{tabular} \\ \hline
Build Trigger                               & None                                                                                                                                  \\ \hline
Archive the artifcats                       & IPSecDiagTool.pdf                                                                                                                     \\ \hline
\end{tabular}
\end{table}

\subsection{Bash Script}
\begin{lstlisting}[language=bash]
#!/bin/sh
pdflatex index.tex > /dev/null 2>&1
bibtex index.aux > /dev/null 2>&1
pdflatex index.tex > /dev/null 2>&1
pdflatex index.tex

mv -f index.pdf IPSecDiagTool.pdf
\end{lstlisting}