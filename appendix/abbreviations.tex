
% \acs prints the longform the first time, after that the short
% using \ac always prints the shortform
   
\chapter*{Abkürzungsverzeichnis}
\addcontentsline{toc}{chapter}{Abkürzungsverzeichnis}
\begin{acronym}[PCAPXX] % [] should contain the longest acronym
    \acro{API}{Application Programming Interface}
    \acro{CI}{Continuous Integration}
    \acro{OOP}{Object Oriented Programming} 
	\acro{ECTS}{European Credit Transfer and Accumulation System}
	\acro{IDE}{Integrated Development Environment}
	\acro{ITA}{Institut für Internet-Technologien und -Anwendungen}
	\acro{JAR}{Java ARchive file}
	\acro{JVM}{Java Virtual Machine}
	\acro{HSR}{Hochschule für Technik Rapperswil}
	\acro{VPS}{Virtual Private Server}
	\acro{PDF}{Portable Document Format}
	\acro{TDD}{Test Driven Delevopment}
	\acro{UI}{User Interface}
	\acro{URL}{Uniform Resource Locator}
	\acro{VM}{Virtual Machine}
	\acro{XML}{Extensible Markup Language}
	\acro{IPsec}{Internet Protocol Security}
	\acro{IP}{Internet Protocol}
	\acro{RUP}{Rational Unified Process}
	\acro{SA}{Semesterarbeit}
	\acro{BA}{Bachelorarbeit}
	\acro{PCAP}{Packet-Capture}
	\acro{MTU}{Maximum Transmission Unit}
	\acro{PMTUD}{Path Maximum Transmission Unit Discovery}
	\acro{ICMP}{Internet Control Message Protocol}
	\acro{ESP}{Encapsulating Security Payload}
	\acro{SPI}{Security Parameters Index}
	\acro{VPN}{Virtual Private Network}
	\acro{IPv4}{Internet Protocol Version 4}
	\acro{IKE}{Internet Key Exchange}
	\acro{OSI}{Open Systems Interconnection}
\end{acronym}
