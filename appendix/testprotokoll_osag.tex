\chapter{Testprotokoll}
\label{chap:Testprotokoll}

\section{Durchführung}
\textbf{Datum:} \\
\textbf{Ort:} \\
\textbf{Getestete Version (GitHub Hash):} \\

\section{Testfälle}

% \Square = leer, \XBox = mit X, \CheckedBox = mit Häckchen
\subsection{Allgemein}
\begin{itemize}
\item[\Square] \textbf{Kompilierung und Deployment:}\\
Das \tool lässt sich auf dem Test-System kompilieren und am erwarteten Speicherort (/opt/ipsecdiagtool/) deployen.
			   
\item[\Square] \textbf{Konfiguration finden:} \\
Das Konfigurationsfile wird am korrekten Ort (/opt/ipsecdiagtool/etc/ipsecdiagtool.conf) erstellt und kann im System gefunden werden. 
			   
\item[\Square] \textbf{Konfiguration anpassen:} \\
Das Konfigurationsfile kann verändert werden und wird beim nächsten Programmstart korrekt eingelesen.
\begin{enumerate} \itemsep1pt \parskip0pt \parsep0pt
  \item Config öffnen, AppID auf 1 setzen.
  \item Das Tool mit dem Debug-Flag starten
  \item Überprüfen ob 1 als AppID ausgegeben wird.
\end{enumerate}

\item[\Square] \textbf{Daemon Installieren:} \\
Das Tool lässt sich als Daemon installieren und starten.
\begin{enumerate} \itemsep1pt \parskip0pt \parsep0pt
  \item Das Tool starten via 'ipsecdiagtool install'.
  \item Überprüfen ob der Daemon verfügbar ist mit 'start ipsecdiagtool', 'status ipsecdiagtool', 'stop ipsecdiagtool'
\end{enumerate}

\item[\Square] \textbf{Daemon Deinstallieren:} \\
Der vom Tool installierte Daemon lässt sich wieder entfernen.
\begin{enumerate} \itemsep1pt \parskip0pt \parsep0pt
  \item Das Tool starten via 'ipsecdiagtool remove'.
  \item Überprüfen ob der Daemon deinstalliert wurde mit 'start ipsecdiagtool'. Wenn ein Fehler kommt wurde er korrekt deinstalliert.
\end{enumerate}
			   
\end{itemize}

\subsection{MTU}
\begin{itemize}
\item[\Square] \textbf{Lokale MTU Discovery:}\\
Sicherstellen das die MTU Discovery lokal korrekt funktioniert.
\begin{enumerate} \itemsep1pt \parskip0pt \parsep0pt
  \item Bestehende Konfigurationsfiles löschen.  
  \item Das Tool als Daemon installieren 'ipsecdiagtool install' und starten 'start ipsecdiagtool'.
  \item Das Tool lokal starten via 'ipsecdiagtool i mtu'.
  \item Das Tool sollte nun die in der Konfiguration eingestellte SnapLen als MTU melden.
\end{enumerate}
			   
\item[\Square] \textbf{MTU Discovery via Tunnel:}\\
Das Tool soll die MTU eines echten IPSec Tunnels ermitteln.
\begin{enumerate} \itemsep1pt \parskip0pt \parsep0pt
  \item Bestehende Konfigurationsfiles neu erstellen oder anpassen so dass folgende Einstellungen konfiguriert sind: SyslogServer, SourceIP, DestinationIP. Die restlichen Einstellungen könne auf den Default Einträgen belassen werden.
  \item Das Tool an beiden Enden des Tunnels als Daemon installieren 'ipsecdiagtool install' und starten 'start ipsecdiagtool'.
  \item Auf einer Seite des Tunnels die MTU Discovery mit dem lokalen Kommando 'ipsecdiagtool mtu' starten.
  \item Die MTU Discovery wird nun via Tunnel durchgeführt. Das erfolgreiche Resultat sollte auf dem Syslog-Server ersichtlich sein.
\end{enumerate}
		

\end{itemize}
\subsection{Packetloss}
