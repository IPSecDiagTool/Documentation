\chapter{Installation Guide}


\section{Systemvoraussetzungen}
Diese Anleitung bezieht sich auf Ubuntu 14.04 x64. Für andere System sollte man sich auf die offizielle Installationsanleitung im GitHub-Repository von goProbe verlassen.

\section{Installation}
Zuerst müssen die fehlenden Dependencies installiert werden. Unter anderem "build-essential, flex, curl und bison".

\begin{lstlisting}[language=bash]
  $ sudo apt-get install build-essential bison flex curl
\end{lstlisting}

Danach clonen wir das GitHub-Repository.

\begin{lstlisting}[language=bash]
  $ git clone https://github.com/open-ch/goProbe.git
\end{lstlisting}

Nun wechseln wir in das goProbe Verzeichnis, erlangen root-Rechte und führen das MAKE-File mit dem Parameter "all" aus.

\begin{lstlisting}[language=bash]
  $ cd ../git/goProbe
  $ sudo su
  $ make all
\end{lstlisting}

Das Make File lädt automatisch die noch fehlenden Dependencies herunter und versucht goProbe zu builden. Wenn der Kompilier-Vorgang fertig läuft dann wurde goProbe erfolgreich installiert. Bei uns hat dass jedoch nicht geklappt, wir mussten hier ins libpcap-Verzeichnis wechseln und dieses zuerst einzeln builden und installieren.

\begin{lstlisting}[language=bash]
  $ cd libpcap-1.5.3
  $ make clean
  $ make install
\end{lstlisting}

Nun können wir zurück ins Hauptverzeichnis wechseln und goProber erneut builden.

\begin{lstlisting}[language=bash]
  $ cd ..
  $ make all
\end{lstlisting}

Der Build-Vorgang zeigt nun zwar einige Build-Warnings sollte aber vollständig durchlaufen. Als letzten Schritt müssen wir jetzt nur noch das Bibliotheken-Verzeichnis "/usr/local/goProbe/lib" korrekt verlinken.

\begin{lstlisting}[language=bash]
  $ cd ~
  $ vim .bashrc
  
  #Am Ende des Files einfuegen:
  export LD_LIBRARY_PATH=/usr/local/goProbe/lib
\end{lstlisting}

Nun kann ein neues Terminal geöffnet werden und der Befehl 'goProbe' sollte ohne Errors ausgeführt werden können.