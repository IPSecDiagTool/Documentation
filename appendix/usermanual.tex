\chapter{Usermanual}
\label{chap:Usermanual}
The \entool{} is a diagnosis application for the continuous monitoring of \acs{IPsec} \acs{VPN} tunnels. It has two main features. Firstly it's capable of passively detecting packet loss occurring within the \acs{IPsec} tunnels. If the packet loss exceeds a specified threshold a Syslog warning is automatically dispatched. Secondly the \entool{} can actively determine the exact \acs{MTU} within a tunnel. This is useful when you're dealing with badly configured routers that block regular \acs{ICMP} messages outside of the tunnel. The \entool{} is designed to run as a daemon/service, but for testing purposes it also has a interactive mode.

\section{Installation Guide}

\subsection{System Requirements}
The \entool{} is currently set up to run on Linux based systems and it requires a installation of the libpcap0.8-dev package. Depending on your local settings you may need to run the \entool{} as superuser (sudo) to be able to capture packets.

Although the \entool{} has only been tested on Linux based systems you may be able to get it running on OS X or windows without or with very few changes to its source code. The application was designed with a broad compatibility in mind.
\cleardoublepage

\section{Installation}
To install the \entool{} you need to run the following bash script. It will take care of setting up a temporary go-environment and getting the proper dependencies. You can also get the latest version of the build-script from the Github-Repository (\url{https://github.com/ipsecdiagtool/ipsecdiagtool/blob/master/build.sh}).

\lstset{language=bash, breaklines=true}
\begin{lstlisting}
#!/usr/bin/env bash
# Title:  build.sh                                       
# URL:    https://github.com/ipsecDiagTool/ipsecDiagTool
# Author: Jan Balmer, Theo Winter
#    
# This script can be used to build IPsecDiagTool in a
# Linux environment. Tested on Ubuntu 14.04.1 LTS.
#
# Dependencies:
#  - libpcap0.8-dev

echo "Cleaning workspace"
if [ -d go ]; then
	rm -rf go1.4.2.linux-amd64.tar.gz
	rm -rf go
    rm -rf workspace
fi

echo "Debug Infos"
go version
go env

echo "Setting up Go"
wget https://storage.googleapis.com/golang
/go1.4.2.linux-amd64.tar.gz
tar xf go1.4.2.linux-amd64.tar.gz
mkdir workspace

export GOROOT=$(pwd)/go
export GOPATH=$(pwd)/workspace
export PATH="$PATH:$GOROOT/bin:$GOPATH/bin"

cd workspace

#Can be deactivated if libpcap is already installed.
echo "Downloading dependencies"
sudo apt-get install libpcap0.8-dev

echo "Getting and building IPsecDiagTool"
go get github.com/ipsecdiagtool/ipsecdiagtool
\end{lstlisting}


\section{Usage}

\subsection{Commands}
The \entool{} can be run either as a local application or in daemon mode. The following commands can be used to control the \entool{} when it is launched as a local application. After this table you will find specific information and examples for typical use cases.

\begin{tabularx}{\textwidth}{l|l|>{\raggedright\arraybackslash}X} 
\textbf{Command} & \textbf{Alternative} & \textbf{Explanation} \\
\hline
install &  & Install the \entool{} as a daemon.\\
uninstall & remove & Remove the \entool{}'s daemon.\\
i mtu &  & Run the mtu discovery in the interactive mode. The daemon does not need to be installed for this mode.\\
i pl [path ]&  & Run the packet loss detection in the interactive mode. The daemon does not need to be installed for this mode. The last argument is optional. You can specify a pcap file for testing purposes, or leave it blank and the network interface specified in the configuration will be used.\\
mtu [srcIP] [dstIP] & mtu-discovery & Tell a running daemon to start detecting the MTU for all configured tunnels. The \entool{} accepts a optional source and destination ip if you want to target a daemon running on a remote computer. \\
debug &  & Show the path to the configuration file, the applicationID and other information useful for debugging.\\
about & version & Display general information about the \entool{}. \\
help & & A list of commands and examples on how to use them.
\end{tabularx}

\subsection{Daemon Installation}
To install the \entool{} as a daemon you need to launch it as superuser via \code{ipsecdiagtool install}. Then a list of potentially supported init-systems for your operating system is displayed. For the Linux operating system you are typically given the options in the listing below.

\begin{lstlisting}[language=bash, caption=Supported linux init-systems]
Please choose one of the following init-systems:
  0. linux-systemd
  1. linux-upstart
  2. unix-systemv
Enter the number of the service you wish to install
\end{lstlisting}

If you are using Ubuntu, choose \code{linux-upstart}, otherwise go with \code{unix-systemv}. Please be aware that \code{linux-systemd},\code{darwin-launchd} and \code{windows-service} have not been tested as part of this thesis but they should potentially work just as well.

It is not recommended to install the \entool{} with multiple init-systems at the same time.

\subsection{Daemon Uninstall}
To uninstall the daemon of \entool{} you should first stop the tool via \code{service ipsecdiagtool stop}. Now you can run \code{ipsecdiagtool uninstall} as superuser and then choose the daemon you wish to remove. You will get a message whether or not the removal was successful. Alternatvely you can go to the location where the daemon init scripts are stored on your operating system and manually remove it. On Ubuntu this is \code{/etc/init.d/} for \code{linux-systemv} or \code{/etc/init/} for \code{linux-upstart}.

\subsection{Daemon Control}
A successfully installed daemon can be controlled via the following simple commands:

\begin{enumerate}
  \item \code{service ipsecdiagtool start}: Start the daemon.
  \item \code{service ipsecdiagtool status}: Shows whether the daemon is running or not.
  \item \code{service ipsecdiagtool stop}: Stops the daemon.
\end{enumerate}

\subsection{Daemon MTU Discovery}
To launch a \acs{MTU} discovery process on the daemon you need to run the local application via \code{ipsecdiagtool mtu}. If you want to start the \acs{MTU} discovery process on a remote machine you can do so by specifying an optional source and destination ip address. For example \code{ipsecdiagtool mtu 192.168.0.10 192.168.0.11} would launch a mtu discovery process on the remote machine \code{192.168.0.11}.

It is not advisable to start a mtu discovery process multiple times without waiting for a running process to finish. The application will not crash but the result of the \acs{MTU} discovery may not be valid.

Whenever a \acs{MTU} discovery process is started a message is sent to the syslog server specified in the configuration. Another message containing the result is sent when the discovery process is finished.

\subsection{Daemon Packet Loss Detection}
There are no additional steps involved to detect packet loss. Whenever the daemon is running, it is automatically detecting packet loss. The results of the packet loss detection are sent to the syslog server specified in the configuration and also written to a file located at \code{/var/log/}.

\subsection{Interactive MTU Discovery}
In order to launch a interactive \acs{MTU} discovery you need to run the command \code{ipsecdiagtool i mtu}. This will start the \acs{MTU} discovery for all configured tunnels and display a live result in the console. A successful discovery process may look like the following listing.

\begin{lstlisting}[language=bash, caption=Successful MTU Discovery]
2015/05/27 00:47:57 MTU Discovery package initialized.
2015/05/27 00:47:57 Interactive testing
Syslog Info: Starting MTU Discovery 1/1 between 192.168.0.10 and 192.168.0.11. Reported by AppID 97691.
2015/05/27 00:48:07 ---------------------------------------------------
2015/05/27 00:48:07 ChanID 0 Range: 0 - 2000   itStep: 100   Timeout: 10ns
2015/05/27 00:48:07 Largest successful packet: 1500
2015/05/27 00:48:07 Received: 100 200 300 400 500 600 700 800 900 1000 1100 1200 1300 1400 1500 
2015/05/27 00:48:07 Missing: 0 1600 1700 1800 1900 2000 
2015/05/27 00:48:17 ---------------------------------------------------
2015/05/27 00:48:17 ChanID 0 Range: 1500 - 1600   itStep: 5   Timeout: 10ns
2015/05/27 00:48:17 Largest successful packet: 1500
2015/05/27 00:48:17 Received: 1500 
2015/05/27 00:48:17 Missing: 1505 1510 1515 1520 1525 1530 1535 1540 1545 1550 1555 1560 1565 1570 1575 1580 1585 1590 1595 1600 
2015/05/27 00:48:27 ---------------------------------------------------
2015/05/27 00:48:27 ChanID 0 Range: 1500 - 1505   itStep: 1   Timeout: 10ns
2015/05/27 00:48:27 Largest successful packet: 1500
2015/05/27 00:48:27 Received: 1500 
2015/05/27 00:48:27 Missing: 1501 1502 1503 1504 1505 
Syslog Info: MTU between 127.0.0.1 and 127.0.0.1 is 1500. Reported by AppID 97691.
\end{lstlisting}

\subsection{Interactive Packet Loss Detection}
To start a interactive packet loss detection you can run the tool with \code{ipsecdiagtool i pl}. It will now listen to the configured port and display all incomming packets. If you want to run a specific test you can also specify the location of a pcap file as an additional parameter. For example \code{ipsecdiagtool i pl /local/awesome.pcap}. Since pcap files can be easily manipulated they're ideal to run special test scenarios.

\subsection{Configuration}
The \entool{} stores its configuration as \code{../[execution directory]/etc/ipsecdiagtool.json}. If there is no \code{ipsecdiagtool.json} file at \code{../[execution directory]/etc} the tool will check if there is a configuration file in your current working directory. This behaviour was implemented to allow for a permanent config when launching the program from an \acs{IDE}. If there is no config in your current working directory a new config is created in \code{../[execution directory]/etc} containing default values.
The config is valid json and can be edited through a normal editor such as \enquote{vim} or \enquote{nano}.

\begin{lstlisting}[language=json, caption=Sample configuration]
{
    "ApplicationID": 0,
    "Debug": false,
    "SyslogServer": "localhost:514",
    "PcapSnapLen": 1516,
    "MTUConfList": [
        {
            "SourceIP": "127.0.0.1",
            "DestinationIP": "127.0.0.1",
            "Timeout": 10,
            "MTURangeStart": 0,
            "MTURangeEnd": 2000,
            "ConcurrentPackets": 20
        }
    ],
    "WindowSize": 32,
    "InterfaceName": "any",
    "AlertTime": 60,
    "AlertCounter": 10,
    "PcapFile": "", 
    "CfgVers": 14
}
\end{lstlisting}

The following list describes each configuration option in detail.

\begin{enumerate}
  \item \code{ApplicationID}: If set to 0 a random \code{ApplicationID} is generated on the program start. If it is set to a positive integer other then 0, this integer is used. The \code{ApplicationID} is used to identify the application that sent \acs{ICMP} packets for the \acs{MTU} discovery.
  \item \code{Debug}: When set to \code{true} the \entool{} prints verbose log \& status messages to the console output or if run as daemon to the log file located at \code{/var/log/}. When set to \code{false} only a minimal amount of log messages are printed.
  \item \code{SyslogServer}: The ip and port of the syslog server that is meant to receive packet loss and \acs{MTU} discovery messages. This option is configured in the format \code{"ip:port"}. For example: \code{"192.168.0.10:514"}.
  \item \code{PcapSnapLen}: This option specifies the maximum packet length that can be captured. It can be set to a integer between 0 and 65535. This option can be used to simulate a \acs{MTU} when running the \acs{MTU} discovery via localhost.
  \item \code{MTUConfList}: Contains a list of \code{MTUConf}'s. A \code{MTUConf} contains all necessary options to detect the \acs{MTU} of a single network connection or \acs{VPN} tunnel.
  \begin{enumerate}
	  \item \code{SourceIP}: The ip of the interface from which the \acs{MTU} discovery should be launched.
	  \item \code{DestinationIP}: The ip of the endpoint between which the \acs{MTU} discovery should be executed.
	  \item \code{Timeout}: The time, in seconds, that is waited before sending the next batch of packets.
	  \item \code{MTURangeStart}: A positive integer (incl. 0) that is smaller then the smallest possible \acs{MTU}. The start \& end range are used to detect the \acs{MTU} faster. If you are uncertain, keep the value 0.
	  \item \code{MTURangeEnd}: A positive integer that is larger then the largest possible \acs{MTU}. For example the \acs{MTU} of a normal, physical ethernet connection is 1500, so a good choice would be 2000. If you are using a different protocol than ethernet you may want to look up its maximal \acs{MTU} and put a value greater than that as \code{MTURangeEnd}.
	  \item \code{ConcurrentPackets}: The maximal amount of packets that are sent out in one batch. More concurrent packets increase the speed in which the \acs{MTU} can be detected but they also cause more network traffic. A good value, that has been tested throughly is 20.
  \end{enumerate}
  \item \code{WindowSize}: The size of the ESP window. Default is 32.
  \item \code{InterfaceName}: The interface that is used to capture packets for the packet loss detection and \acs{MTU} discovery. Ideally this option is set to \code{any} in order to capture packets from all available interfaces. 
  \item \code{AlertTime}:  How far back the packets are validated for the alert. The value is in seconds.
  \item \code{AlertCounter}: How many packets have to be missing before an alert is sent to the syslog server.
  \item \code{PcapFile}: The location of a pcap file. This pcap file is used for the packet loss detection instead of real network traffic from a interface. If this is empty the real network interface is used.
  \item \code{CfgVers}: There is never any need for you to manually change this value. The CfgVers is used to determine whether the current configuration needs to be updated automatically when a new configuration-option has been added to the application.
\end{enumerate}

