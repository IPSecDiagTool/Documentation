\chapter{Usermanual}
\label{chap:Usermanual}
The \tool is a diagnosis application for the continuous monitoring of \acs{IPSec} \acs{VPN} tunnels. It has two main features. Firstly it's capable of passively detecting packet loss occurring within the \acs{IPSec} tunnels. If the packet loss exceeds a specified threshold a Syslog warning is automatically dispatched. Secondly IPSecDiagTool can actively determine the exact \acs{MTU} within a tunnel. This is useful when you're dealing with badly configured routers that block regular \acs{ICMP} messages outside of the tunnel. The \tool is designed to run as a daemon/service, but for testing purposes it also has a interactive mode.

\section{Installation Guide}
This chapter explains how to install the \tool.

\subsection{System Requirements}
The \tool is currently set up to run on Linux based systems and it requires a installation of the libpcap0.8-dev package. Depending on your local settings you may need to run the \tool as superuser (sudo) to be able to capture packets.

Although the \tool has only been tested on Linux based systems you may be able to get it running on OS X or windows without or with very few changes to its source code. The application was designed with a broad compatibility in mind.

While capturing 300mbit/s of packet-data the tool requires about 20\% CPU Usage on a modern Intel Xeon processor. So make sure to have enough spare processing power on your VPN routers.

\todo{CPU and RAM performance impact requirements überprüfen / mit aktuellen Daten ergänzen.}

\section{Installation}
To install the \tool you need to run the following bash script. It will take care of setting up a tempoary go-environment and getting the proper dependencies. You can also get the latest version of the build-script from the Github-Repository (\url{https://github.com/ipsecdiagtool/ipsecdiagtool/blob/master/build.sh}).

\lstset{language=bash, breaklines=true}
\begin{lstlisting}
#!/usr/bin/env bash
# Title:  build.sh                                       
# URL:    https://github.com/IPSecDiagTool/IPSecDiagTool
# Author: Jan Balmer, Theo Winter
#    
# This script can be used to build IPSecDiagTool in a
# Linux environment. Tested on Ubuntu 14.04.1 LTS.
#
# Dependencies:
#  - libpcap0.8-dev

echo "Cleaning workspace"
if [ -d go ]; then
	rm -rf go1.4.2.linux-amd64.tar.gz
	rm -rf go
    rm -rf workspace
fi

echo "Debug Infos"
go version
go env

echo "Setting up Go"
wget https://storage.googleapis.com/golang
/go1.4.2.linux-amd64.tar.gz
tar xf go1.4.2.linux-amd64.tar.gz
mkdir workspace

export GOROOT=$(pwd)/go
export GOPATH=$(pwd)/workspace
export PATH="$PATH:$GOROOT/bin:$GOPATH/bin"

cd workspace

#Can be deactivated if libpcap is already installed.
echo "Downloading dependencies"
sudo apt-get install libpcap0.8-dev

echo "Getting and building ipsecdiagtool"
go get github.com/ipsecdiagtool/ipsecdiagtool
\end{lstlisting}

\todo{Make sure to update or use direct code-link !!}

\section{Usage}

\subsection{Commands}
The following commands can be used to control the \tool.

\begin{tabularx}{\textwidth}{l|l|>{\raggedright\arraybackslash}X} 
\textbf{Command} & \textbf{Alternative} & \textbf{Explanation} \\
\hline
install &  & Install the \tool as a daemon.\\
uninstall & remove & Removes the \tool daemon.\\
interactive & demo & Allows for interactive testing. Results are directly printed to the console. The daemon is not used. \\
mtu-discovery & mtu & Tells the daemon to start finding the MTU for all configured tunnels. \\
about & version & Displays general information about the \tool. \\
help & & A list of commands and how to use them.
\end{tabularx}