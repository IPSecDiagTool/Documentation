\section{Einleitung}
\label{sec:Einleitung}

Dieses Kapitel beschreibt die spezifischen Anforderungen an das \tool, welche in Zusammenarbeit mit der \osag erarbeitet wurden.

\subsection{Produktfunktion}
Mit dem Tool soll das Erkennen und Diagnostizieren von Problemen bei \acs{IPSec} \acs{VPN} Verbindungen vereinfacht werden. Dabei soll das \tool die Möglichkeit bieten passiv Paket-Verluste zu erkennen sowie aktiv die \acs{MTU} zu bestimmen. Die Applikation ist als konstant laufender Service konzipiert und bietet daher keine graphische Oberfläche sondern nur ein Commandline-Interface. Für die Kommunikation mit den Benutzern des Tools sollen Log-Einträge verwendet werden.

\subsection{Benutzercharakteristik}
Die Benutzer des \tool sind Netzwerkadministratoren und Entwickler die \acs{IPSec} \acs{VPN} Tunnels betreiben. Daher kann ein gutes Mass an technischem Verständnis und Erfahrung bei der Verwendung von Commandline-Applikationen gerechnet werden.

\subsection{Abhängigkeiten}
Das \tool ist grundsätzlich eine eigenständige Applikation. Es wurde jedoch bereits in der Ausschreibung dieser Arbeit festgelegt dass libpcap als Library für das Capturen von Paketen verwendet werden soll. Daher ist das \tool von libpcap und einem entsprechenden Wrapper abhängig.