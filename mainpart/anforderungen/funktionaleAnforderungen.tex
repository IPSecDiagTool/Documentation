\section{Funktionale Anforderungen}
\label{sec:Funktionale Anforderungen}

\subsection{Muss-Kriterien}

\begin{itemize}

\item \textbf{M0: Paket-Verlust messen} \\
Das \tool{} muss den Paket-Verlust von allen konfigurierten \ac{IPsec} \ac{VPN}-Verbindungen bestimmen können.

\item \textbf{M1: Paket-Verlust melden} \\
Wenn ein konfigurierter Grenzwert überschritten wird muss das \tool{} automatisch eine Meldung an einen zentralen Syslog Server absetzen.

\item \textbf{M2: Paket-Verluste dokumentieren} \\
Für jedes verlorene Paket muss in einem lokalen Log ein Eintrag erzeugt werden. Wenn das Paket später noch ankommt wird dies ebenfalls vermerkt.

\item \textbf{M3: MTU Bestimmen} \\
Das \tool{} muss in der Lage sein die \ac{MTU} zwischen zwei Routern bestimmen zu können. Dabei läuft das \tool{} auf mindestens einem der beiden Routern.

\item \textbf{M4: MTU Melden} \\
Eine erfolgreich bestimmte \ac{MTU} muss an einen zentralen Syslog Server gemeldet werden.

\item \textbf{M6: Konfiguration} \\
Alle notwendigen Einstellungen müssen sich vom User ohne Änderung des Source Codes vornehmen lassen.

\end{itemize}

\subsection{Soll-Kriterien}
\begin{itemize}

\item \textbf{S0: MTU für mehrere Tunnels bestimmen} \\
Für die \ac{MTU} Bestimmung sind mehrere Tunnels konfigurierbar. Wenn die \ac{MTU} Bestimmung ausgeführt wird, werden die \ac{MTU}s von allen konfigurierten Tunnels bestimmt.

\item \textbf{S1: Parallele MTU Bestimmung} \\
Die \ac{MTU} Bestimmung für mehrere Tunnels soll parallel ablaufen um Zeit zu sparen.

\item \textbf{S2: Paket-Verlust Messung für mehrere Tunnels} \\
Das \tool{} soll in der Lage sein den Paket-Verlust für alle Tunnels, die auf dem Rechner laufen, zu bestimmen.
  	
\item \textbf{S3: Daemon Modus} \\
Um langfristig den Paket-Verlust messen zu können soll das Tool in der Lage sein als Unix-Daemon zu laufen.

\end{itemize}

\subsection{Kann-Kriterien}
\begin{itemize}

\item \textbf{K0: Statistiken erstellen} \\
Es kann direkt aus dem \tool{} eine Statistik zum Paket-Verlust erstellt werden.

\item \textbf{K1: Solo \ac{MTU} Bestimmung} \\
Für die \ac{MTU} Bestimmung ist nur eine \tool{} Installation notwendig. Der Router auf der Gegenseite muss nicht speziell konfiguriert sein.

\end{itemize}