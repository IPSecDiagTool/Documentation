\section{Use Cases}
\label{sec:Use Cases}

\subsection{Aktoren \& Stakeholder}

\begin{table}[H]
\begin{tabularx}{\textwidth}{l|>{\raggedright\arraybackslash}X} 
\textbf{Aktoren \& Stakeholder} & \textbf{Interessen} \\
\hline
Netzwerk Administrator & Möchte einen \ac{IPsec} Tunnel betreiben der kein Paket Verlust hat und keine Fragmentierungsprobleme hat. Wen dass nicht möglich ist möchte er zumindest frühzeitig über solche Probleme informiert werden, so dass diese behoben werden können bevor ein Kunde eine Beschwerde einreichen muss. \\
System (\tool{}) & Soll den Netzwerk Administrator bei Paket Verlusten oder Fragmentierungsproblemen frühzeitig informieren können. \\
\end{tabularx}
\caption{Aktoren \& Stakeholder}
\end{table}

\subsection{UC1: Bestimmen von Paketverlusten}
Der User soll in der Lage sein mit dem \tool{} passiv den Paket-Verlust zu ermitteln. Der ermittelte Paket-Verlust soll sowohl Prozentual als auch mit einer konkreten Zahl an verlorenen Paketen pro Zeiteinheit ausgewiesen werden. Die Zeit, in der der prozentuale Paket-Verlust bestimmt wird, soll durch den User konfigurierbar sein.

\subsection{UC1.1: Auslösen von Alerts bei überschrittenem Grenzwert}
Wenn ein bestimmter Grenzwert beim Überprüfen von Paket-Verlusten überschritten wird, soll eine Warnung ausgelöst werden. Diese Warnung soll sowohl an einen zentralen Server übermittelt als auch lokal in einem in einem Log-File gespeichert werden. Der Grenzwert soll konfigurierbar sein.

\subsection{UC2: MTU Bestimmung}
Ein User des \tool{}s soll die Möglichkeit haben die \ac{MTU} innerhalb eines Tunnels zwischen zwei Routern zu bestimmen.

\subsection{UC2.1: MTU Bestimmung für mehrere Tunnels}
Wenn auf einem Router mehrere Tunnels installiert sind soll der User die Möglichkeit haben alle diese Tunnels im \tool{} zu konfigurieren. Die \ac{MTU} Bestimmung soll dann auch für jeden Tunnel gleichzeitig durchgeführt werden so dass der User nicht unverhältnismässig lange auf ein Resultat warten muss.

\subsection{UC2.2: Periodische Bestimmung der MTU}
Die \ac{MTU} Bestimmunsoll programmatisch, periodisch durchgeführt werden können.

\subsection{UC3: Permanente Ausführung als Service}
Das \tool{} soll vom User als Service, der permanent auf einem Router läuft, gestartet werden können.

\subsection{UC4: Konfiguration}
Beim ersten Start des \tool{}s soll automatisch ein Konfigurationsfile erstellt werden. In diesem File sollen die für die \ac{MTU} Bestimmung und das Detektieren von Paket Verlusten notwendigen Einstellungen gespeichert werden. Bei der automatischen Erstellung sollen sinnvolle Werte gesetzt werden die eine Demonstration des \tool{}s erlauben.

\subsection{Optional UC5: Passiv die IKE Header untersuchen}
Das \tool{} soll die Möglichkeit bieten passiv die \ac{IKE} Header zu untersuchen und dabei Probleme festzustellen. Dieser Use Case wird nur bearbeitet wenn alle anderen Use Cases erfüllt sind. In so einem Fall müsste der UC5 noch genauer spezifiziert werden.