\section{Use Cases}
\label{sec:Use Cases}

\subsection{Aktoren \& Stakeholder}

\begin{table}
\begin{tabularx}{\textwidth}{l|>{\raggedright\arraybackslash}X} 
\textbf{Aktoren \& Stakeholder} & \textbf{Interessen} \\
\hline
Netzwerk Administrator & Möchte einen IPsec Tunnel betreiben der kein Paket Verlust hat und keine Fragmentierungsprobleme hat. Wen dass nicht möglich ist möchte er zumindest frühzeitig über solche Probleme informiert werden, so dass diese behoben werden können bevor ein Kunde eine Beschwerde einreichen muss. \\
VPN Benutzer & Möchte eine schnelle und stabile \acs{VPN} Verbindung benutzen. Bei Problemen mit der Verbindung möchte er eine möglichst schnelle Lösung des Problems haben. \\
System (\tool) & Soll den Netzwerk Administrator bei Paket Verlusten oder Fragmentierungsproblemen frühzeitig informieren können. \\
\end{tabularx}
\caption{Aktoren \& Stakeholder}
\end{table}

\subsection{UC1: Bestimmen von Paketverlusten}
Der Paketverlust soll passiv, durch auslesen der ESP-Sequenznummern, vom \tool ermittelt werden. Der Paketverlust soll sowohl Prozentual als auch mit einer konkreten Zahl an verlorenen Paketen pro Zeiteinheit ausgewiesen werden. Die Zeiteinheit soll konfigurierbar sein.

\subsection{UC1.1: Auslösen von Alerts bei überschrittenem Grenzwert}
Wenn ein vorbestimmter Grenzwert beim Überprüfen des Paket-Verlust  überschritten wird soll ein Alert ausgelöst werden. Dieser Alert kann in der Form eines Logfiles sein oder als Nachricht an einen Syslog-Server. Der Grenzwert soll konfigurierbar sein.

\subsection{UC2: Bestimmen der MTU}
Durch einen Tunnel werden testweise Pakete unterschiedlicher Grösse und einem "Don't fragment"-Flag gesendet und auf der anderen Seite wieder aufgezeichnet. So wird die MTU unabhängig von PathMTU im Tunnel selbst ermittelt. Anhand der Resultat von UC2 soll es möglich sein Fragmentierungsprobleme zu erkennen.

\subsection{UC2.1: Periodische Bestimmung der idealen MTU}
Alle Tunnels sollen periodisch, automatisch vom \tool analysiert werden. Dabei wird die ideale MTU wie in UC1 beschrieben bestimmt. Dies kann mittels eines integrierten Daemon oder Cron-Jobs realisiert werden.

\subsection{UC3: Konfiguration}
Das \tool soll grundsätzliche den Verkehr von allen Tunnels aufzeichnen. Es soll jedoch auch möglich sein anhand eines Konfigurationsfiles nur den Verkehr eines spezifischen Tunnels aufzuzeichnen. Dies wird durch das konfigurieren einer Source- und Destination Adresse erreicht.

\subsection{Optional UC4: Passiv die IKE Header untersuchen}
Das \tool bietet die Möglichkeit passiv die IKE Header zu untersuchen und dabei Probleme festzustellen. Dieser Use Case wird nur bearbeitet wenn alle anderen Use Cases erfüllt sind und es müsste dann noch genauer festgelegt werden welche Probleme erkannt werden können.