\section{Use Cases}
\label{sec:Use Cases}


\subsection{Aktoren \& Stakeholder}


\subsection{UC1: Bestimmen von Paketverlusten}
Der Paketverlust soll passiv, durch auslesen der ESP-Sequenznummern, vom \tool ermittelt werden.

\subsection{UC1.1: Auslösen von Alerts bei überschrittenem Grenzwert}
Wenn ein vorbestimmter Grenzwert beim Überprüfen des Paket-Verlust  überschritten wird soll ein Alert ausgelöst werden. Dieser Alert kann in der Form eines Logfiles sein.

\subsection{UC2: Bestimmen der optimalen MTU}
Durch einen Tunnel werden testweise Pakete unterschiedlicher Grösse gesendet und auf der anderen Seite wieder aufgezeichnet. Dabei wird die ideale MTU ermittelt welche zu einer möglichst tiefen Fragmentierung führt.

\subsection{UC2.1: Periodische Bestimmung der idealen MTU}
Alle Tunnels sollen periodisch, automatisch vom \tool analysiert werden. Dabei wird die ideale MTU wie in UC1 beschrieben bestimmt. Dies kann mittels eines integrierten Daemon oder Cron-Jobs realisiert werden.

\subsection{UC3: Konfiguration}
Das \tool soll grundsätzliche den Verkehr von allen Tunnels aufzeichnen. Es soll jedoch auch möglich sein anhand eines Konfigurationsfiles nur den Verkehr eines spezifischen Tunnels aufzuzeichnen. Dies wird durch das konfigurieren einer Source- und Destination Adresse erreicht.

\subsection{Optional UC4: Passiv die IKE Header untersuchen}
Todo.. tritt nur ein wenn vorige UC's vollständig erfüllt wurden.