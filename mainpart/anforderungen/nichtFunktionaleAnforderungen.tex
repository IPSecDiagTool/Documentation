\section{Nichtfunktionale Anforderungen}
\label{sec:Nichtfunktionale Anforderungen}

\subsection{Angemessenheit}
\begin{itemize}
	\item Die Applikation soll schlank und übersichtlich und in einem einheitlichen Stil programmiert werden.
	\item Es sollen möglichst alle der genannten Anforderungen erfüllt werden.
\end{itemize}

\subsubsection{Funktionalität}
\begin{itemize}
\item Das Zeitfenster der Überwachung von Paketverlusten ist konfigurierbar.
\item Das Diagnose Tool funktioniert mindestens bis zu einer Datenrate von 300Mbit/s.
\item Für die Überprüfung der \acs{MTU} fungiert eine Seite als Client und die andere als Server, wobei die gleiche Software verwendet werden kann.
\item Es ist konfigurierbar welches Netzwerkinterface bei der Überwachung verwendet werden soll.
\end{itemize}

\subsubsection{Testbarkeit}
\begin{itemize}
\item Businesslogik muss über alle Use Cases (Normal verhalten) mit automatischen Unit Test abgedeckt sein.
\end{itemize}

\subsubsection{Fehlertoleranz}
\begin{itemize}
\item Installation des Tools kann automatisch überprüft werden.
\item Das System muss nach einem Neustart in einem definierten Zustand sein.
\item Es wird vor dem Start überprüft ob die eingestellte Konfiguration gültig ist.
\end{itemize}

\subsubsection{Änderbarkeit}
\begin{itemize}
\item Daten und Reports werden in einem Standard gespeichert, welcher auch von anderen Systemen gelesen werden kann (CSV, XML)
\item Der Code inklusive Kommentar sind auf englisch um eine leichtere Anpassbarkeit zu erreichen.
\end{itemize}

\subsubsection{Anforderungen an Umgebung}
\begin{itemize}
\item Das Tool ist unter dem Betriebssystem Linux lauffähig, es werden jedoch root Rechte benötigt.
\item Für die Umgebung braucht es die \acs{PCAP} Library sowie die Go Programming Language.
\item Das Tool soll Open-Source sein und der Quellcode daher für jeden zugänglich.
\end{itemize}