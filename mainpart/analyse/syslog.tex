\section{Syslogprotokoll}
\label{sec:Syslogprotokoll}

Syslog ist eines der am häufigsten eingesetzten Protokolle zum Übermitteln von Log-Meldungen im Netzwerk. Durch den einfachen Aufbau und die breite Unterstützung ist es leicht einsetzbar.


\subsection{ Aufbau}

Eine Syslog-Meldung besitzt ein Severity-Field, womit der Schweregrad einer Nachricht festgelegt werden kann.

\noindent Die Werte besitzen folgende Bedeutungen:

\begin{table}[H]
\begin{tabular}{|p{0.5in}|p{3.7in}|} \hline 
Code & \textbf{Severity} \\ \hline 
0 & \textbf{Emergency:} system is unusable \\ \hline 
1 & \textbf{Alert:} action must be taken immediately \\ \hline 
2 & \textbf{Critical:} critical conditions \\ \hline 
3 & \textbf{Error:} error conditions \\ \hline 
4 & \textbf{Warning:} warning conditions \\ \hline 
5 & \textbf{Notice:} normal but significant condition \\ \hline 
6 & \textbf{Informational:} informational messages \\ \hline 
7 & \textbf{Debug:} debug-level messages \\ \hline 
\end{tabular}
\caption{Severity-Feld einer Syslogmeldung nach RFC 5424}
\end{table}

Zudem besteht eine Meldung noch aus einem Header, sowie der eigentlichen Nachricht.