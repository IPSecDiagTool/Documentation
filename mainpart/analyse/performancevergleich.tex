\section{Performance Vergleich}
\label{sec:Performance Vergleich}

\subsection{Testaufbau}
Zwei Desktop-Rechner der HSR, mit je 16GB Ram und Intel Xeon 3.4Ghz Quad-Core CPUs, sind via Gigabit-Lan miteinander verbunden. Auf den Rechnern läuft Ubuntu 14.04 x64 sowie jPerf und die jeweils getestete Software.

%todo kontrollieren das Quadcore und nicht Ocata-Core.
%todo Grafik mit MS Visio die unser Testaufbau darstellt.

\subsection{Testdurchführung}
Auf einem der beiden Rechnern läuft jeweils jPerf im Server-Modus sowie die getestete Software. Auf dem anderen Computer läuft jPerf im Client-Modus.
Via. jPerf wird nun soviel Traffic erzeugt um die 1Gbit/s-Leitung möglichst stark auszulasten, d.h. durchschnittlich 900mbit/s. Die getestete Software zeichnet dabei die ganzen Pakete auf und sollte dabei 300mbit/s an Traffic ertragen können. 300mbit/s sind gem. Open Systems AG die Lastspitzen mit denen etwa zu rechnen sind.

\subsection{Ergebnisse}
Java und Golang sind von den Ergebnissen her recht ähnlich. Beide haben die Anforderung von 300mbit/s erfolgreich erfüllt. Golang ist mit den Durchschnittlich 17\% CPU Auslastung etwas performanter als Java. Die 31\% CPU Lastspitze bei Java gibt es jeweils nur wenn das Programm zum ersten Mal gestartet wird und kommt daher dass dann die ganze \acs{JVM} zuerst hochgefahren werden muss.
Beim Memory siehts bei goProbe klar besser aus weil es den ganzen \acs{JVM} Overhead nicht hat.

\begin{table}[h]
\begin{tabular}{|l|l|l|l|l|l|l|}
\hline
\rowcolor[HTML]{C0C0C0} 
\textbf{Software} & \textbf{CPU Top} & \textbf{CPU Ø} & \textbf{Mem Ø} & \textbf{VIRT\footnotemark[1]} & \textbf{RES\footnotemark[2]} & \textbf{SHR\footnotemark[3]} \\ \hline
JNetPcap          & 31\%             & 20\%           & 0.9\%          & 7030296       & 147840       & 19588        \\ \hline
goProbe           & 18\%             & 16\%           & 0.3\%          & 315268        & 1964         & 1676         \\ \hline
\end{tabular}
\end{table}

\footnotetext[1]{VIRT steht für die virtuelle Grösse eines Prozesses.}
\footnotetext[2]{RES steht für den tatsächlich, physisch verbrauchten Hauptspeicher.}
\footnotetext[3]{SHR zeigt wie viel von VIRT mit anderen Prozessen teilbar ist. Dazu gehören z.B. Shared Libraries.}