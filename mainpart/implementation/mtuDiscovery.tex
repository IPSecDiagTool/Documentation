\section{MTU Discovery}
\label{sec:MTU Discovery}

Die \ac{MTU} Discovery wurde in der \code{mtu} Package implementiert. Darin enthalten sind \code{analyze.go}, \code{capture.go} und \code{send.go} sowie die dazugehörigen Unit Tests.

\subsection{Öffentliche Funktionen}
Die \code{mtu} Package bietet die folgenden öffentlichen Funktionen:

\begin{itemize}
\item \textbf{Init(config config.Config, icmpPackets chan gopacket.Packet):} \\
Diese Funktion wird zur Initialisierung der \code{mtu} Package verwendet. Die Package muss zwingend vor der Verwendung der \code{FindAll()} Funktion initialisiert werden. Zur Initialisierung wird die Konfiguration sowie die der für \ac{ICMP} Pakete verwendete Go-Channel übergeben.
\item \textbf{Find(mtuConf config.MTUConfig, appID int, chanID int, mtuOK chan int, wg *sync.WaitGroup) int:} \\
Die \code{Find(..)} Funktion sucht die \ac{MTU} einer Verbindung und gibt diese dann zurück. Diese Funktion wurde im Hinblick auf eine potentielle Verwendung als Library veröffentlicht. Aktuell benötigt sie aber etwas viele Parameter und sollte refactored werden. Im \tool{} selbst wird die Funktion nur in der \code{MTU} Package gebraucht.
\item \textbf{FindAll():} \\
\code{FindAll()} sucht die \ac{MTU}s aller konfigurierten Verbindungen und meldet die Ergebnisse via der \code{logging} Package.
\item \textbf{RequestDaemonMTU(appID int, sourceIP string, destinationIP string):} \\
Die Funktion \code{RequestDaemonMTU(..)} sendet ein ICMP Request an die übergebene Destionation IP mit dem Kommando einen \ac{MTU} Discovery Vorgang zu starten. Die Funktion erwartet die AppID, Source- und Destination Adresse.
\end{itemize}

\subsection{Implementation des FastMTU Algorithmus}
Der FastMTU Algorithmus ist in \code{analyze.go} implementiert. 


\subsection{ICMP Paket Payload}
Alle vom \tool{} verschickten \acs{ICMP} Pakete haben jeweils die folgende Payload:

\begin{itemize}
  \item \textbf{AppID:} Eindeutige ID des aktiven \tool{}. Wird verwendet um mehrere gleichzeitig laufende \tool{} auf einem Rechner zu unterscheiden. Die AppID wird entweder in der Konfiguration fix festgelegt oder auf 0 gesetzt. Wenn die AppID in der Konfiguration auf 0 gesetzt wurde dann wird beim Programmstart eine zufällige AppID generiert.
  \item \textbf{ChannelID:} ID des GO-Channels von dem das Paket versendet wurde. Wird benötigt um für mehrere Tunnels gleichzeitig die \acs{MTU} festzustellen zu können. Die ChannelID wird jeweils beim Start eines \acs{MTU} Discovery Vorgangs zugeteilt.
  \item \textbf{Command:} Die eigentliche Nachricht des Pakets. Wird verwendet um sicherzustellen dass dieses \acs{ICMP} Paket wirklich von einem \tool{} versendet wurde und nicht ein sonstiges \acs{ICMP} Paket.
  \item \textbf{Null-Array:} Ein mit Nullen gefülltes Array von variabler Grösse. Wird verwendet um dem Paket seine vorbestimmte Grösse zu geben.
\end{itemize}

