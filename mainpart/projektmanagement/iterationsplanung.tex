\section{Iterationen und Meilensteine}
\label{sec:Iterationen und Meilensteine}

\subsection{Iterationsplanung}
Zur Bewältigung dieser Arbeit verwenden wir einen angepassten \acl{RUP} Ablauf. \acs{RUP} definiert 4 Phasen, welche iterativ mehrfach durchlaufen werden.

\begin{itemize}
  \item \textbf{Inception} \newline In der Inception Phase geht es darum die Details des Arbeitsauftrags auszuarbeiten und ein Fundament für das Projekt zu legen. Dies beinhaltet das Aufsetzen der Projekt-Management Software, erstellen von Repositories \& File-Share. Aber auch das wählen des Vorgehensmodells.
  \item \textbf{Elaboration 1} \newline In der ersten Elaboration Phase geht es darum sich für eine Programmiersprache und einen Pcap-Wrapper zu entscheiden. Ausserdem wird der Projektplan erstellt und die Dokumentation aufgesetzt.
  \item \textbf{Elaboration 2} \newline In der zweiten Elaboration Phase erstellen wir das Domain Modell sowie einen groben Prototypen unserer Applikation.
  \item \textbf{Construction 1} \newline In der ersten Construction Phase soll die Hälfte der MUSS-Funktionalität implementiert sein. Dies beinhaltet Use Case 1 und Use Case 2.
  \item \textbf{Construction 2} \newline In der zweiten Construction Phase soll die zweite Hälfte der MUSS-Funktionalität implementiert sein. Dies beinhaltet Sub-Use Case 1 und Sub-Use Case 2.
  \item \textbf{Construction 3} \newline Construction Phase 3 Bietet Pufferzeit falls in Construction Phase 1 \& 2 nicht alle Funktionalität implementiert werden konnte. Für T.Winter stellt diese Phase ausserdem die Transition dar, da die Semesterarbeit am ende dieser Phase zu Ende ist.
  \item \textbf{Transition} \newline In der Transition-Phase sollen noch allfällige Bugs beseitigt werden. Sowie das Usermanual geschrieben und die Präsentation vorbereitet werden. Auch allfällige Rest-Dokumentationsarbeit soll in dieser Phase erledigt werden.
\end{itemize}

\subsection{Meilensteine}

\subsection{Zeitplan}

\subsection{Arbeitspakete}