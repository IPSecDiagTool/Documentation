\section{Einführung}
\label{sec:Einführung}

\subsection{Zweck}
Dieses Dokument dient zur Planung der Semester/Bachelorarbeit "\tool ". Hier werden die Aufgabenstellung, die Iterationsplanung und die Meilensteine definiert.

\subsection{Aufgabenstellung}
Die Firma Open Systems betreibt im Auftrag ihrer Kunden rund um die Uhr ein weltweites Netz von VPN Verbindungen. Dabei steht eine hohe Qualität und Verfügbarkeit der IPSec Tunnels an erster Stelle.
Unter dem Linux-Betriebsystem soll ein Diagnose-Tool für IPSec Verbindungen erstellt werden das folgende Fähigkeiten hat:

Die Programmier- oder Skriptsprache, mit der das Tool erstellt wird ist offen. Es muss aber die Möglichkeit bestehen, die pcapLibrary zum passiven Aufzeichnen der Netzwerkpakete einzubinden.

\begin{itemize}
	\item Passives Bestimmen der IPsec Paketverluste durch Erfassen der laufenden ESP Sequenznummern von ankommenden IPsec Paketen.
	\item Aktive Diagnose von IPsec Fragmentierungsproblemen und Bestimmung der optimalen MTU. Durch einen Tunnel werden IP Pakete variabler Grösse z.B. in der Form von IPMP Requests gesendet und überprüft, ob IP Fragmentierung auftritt und Fragmente verloren gehen.
\end{itemize}

\subsection{Ziele}

\begin{itemize}

  \item Auswahl einer geigneten Programmiersprache und Wrappers für die pcapLibrary.
  \item Einarbeiten in IPSec Tunnels und aufsetzen einer geeigneten Testumgebung.
  \item Spezifikation, Implementation und Test der \tool Funktionalität.

\end{itemize}

\subsection{Abgabetermine}
Der späteste Abgabetermin für die Semesterarbeit ist der Freitag, 29. Mai 2015.
Der späteste Abgabetermin für die Bachelorarbeit ist der Freitag, 12. Juni 2015.
Aufgrund der speziellen Situation, eine Kombination einer Bachelorarbeit und Semesterarbeit, ist es möglich das beide Arbeiten am 12. Juni 2015 eingereicht werden.