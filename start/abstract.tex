% Der Abstract richtet sich an den Spezialisten auf dem entsprechenden Gebiet
% und beschreibt daher in erster Linie die (neuen, eigenen) Ergebnisse und
% Resultate der Arbeit. Es umfasst nie mehr als eine Seite, typisch sogar nur
% etwa 200 Worte (etwa 20 Zeilen). Es sind keine Bilder zu verwenden.

\chapter*{Abstract}\addcontentsline{toc}{chapter}{Abstract}

Das \tool ist eine Commandline-Applikation mit dem Ziel die Fehler-Diagnose bei IPSec VPN Verbindungen zu vereinfachen. Dabei wurden zwei Haupt Use Cases bearbeitet. Zum einen das konstante Detektieren von Paket-Verlusten durch ein passives Capturen von IPSec ESP Paketen. Zum anderen die Möglichkeit eine periodische Überprüfung der MTU innerhalb eines VPN Tunnels durchzuführen. Um die MTU festzustellen wird eine Menge von unterschiedlich grossen ICMP Paketen mit einem "Don't fragment"-Flag durch den Tunnel geschickt. Die erfolgreichen Pakete generieren eine Antwort und so kann die korrekte MTU innert kürzester Zeit festgestellt werden. Das \tool ist so designet dass es als Daemon konstant auf den Routern auf beiden Seiten eines Tunnels läuft. Um den Performance-Impact möglichst gering zu halten wurde bei der Entwicklung auf die Programmiersprache Go gesetzt. \tool setzt für das Paket Capturen auf die Library libpcap sowie den von Google entwickelten Wrapper gopacket. \tool wurde gem. den Anforderungen und im Auftrag der \osag entwickelt. \osag betreibt im Auftrag ihrer Kunden rund um die Uhr ein weltweites Netz von IPSec VPN Tunnels.